\section{Payload Design Overview}
The CubeSat had been pivotal to the space research and development industry and had generally increased our accessibility to the cosmos. The cubesat platform uniquely offers an extremely low construction and launch cost in comparison to major satellite manufacturers. This has spurred on many educational bodies and small research groups to collect and analyse their own data, especially in developing nations~\cite{woeller}. Universities have pioneered the build of the smallest of satellites (nano- and pico-), this has been assisted with the miniaturisation in many technological fields such as electronics, materials ans sensors. These small sizes enable the cubesat to `piggy-back' on the launch of much larger satellites, because of this they are able to get to very high (and expensive) orbits of a fraction of the price. Many big aerospace companies have also made use of the tiny platform such as Orbital Sciences (2006~\cite{heidt}) and Boeing (2009~\cite{straub}) along with the United Nations, who have formally recognised the developmental benefits of small satellites~\cite{rycroft}. \\

\noindent
Despite the quick growth of this industry in the aerosapce and related academia fields, we are only now seeing the cubesat break into early STEM education. Currently, only science and technology scholars and graduates have the full accessibility to the design of cubesats, our ease of access is not well known amongst the general public. Snapsat hopes to change this, by bringing space to social media via beautiful photographs and Twitter. Snapsat is a nano-satellite designed for outreach and space accessibility for educational bodies and the general public. In a sun-synchronous orbit at an altitude at 350km, Snapsat will be in the prime position of Earth observation. Users and sponsors can send a message to the cubesat, which will take a low resolution image of the Earth and tweet it to the world. \\

\noindent
The camera we will be making use a TTL (Through The Lens) camera, which can take snapshots and transmit over a TTL serial link. The image output is a pre-compressed JPEG, making memory storage and transmission easier. This module was also chosen for its ease of integration - requiring only two digital pins, there is an extensive arduino library also available. \\

\noindent
Currently, the final build will result in a balloon launch test. Following the success of this, SnapSat will be launched on a sounding rocket, where is will be placed in a low Earth orbit for a maximum lifetime of three months.