\section{Spacecraft Modes of Operation}
The spacecraft will experience the following modes during its lifetime. A different configuration of system operations and instructions will be executed by \textit{SnapSat} in each case. These are summarised in table~\ref{tab:modesofoperation} below.

\begin{table}[H]
    \centering
    \caption{SnapSat Modes of Operation}
    \vspace{0.15cm}
    \label{tab:modesofoperation}
    {\renewcommand{\arraystretch}{1.4}%
        \begin{tabular}{|>{\arraybackslash}m{4.5cm}|>{\arraybackslash}m{10.5cm}|}
            \hline
            \textbf{Spacecraft Mode} & \textbf{Description} \\ \hline\hline
            Safe mode & (NOTE: this is example text) This mode is intended to keep the satellite alive. Only the essential components are ON all the time - such as the OBC, power board and VHF receiver. Transmitter is turned ON occasionally. 
            Has uncontrolled attitude. 
             \\\hline
            Recovery/De-tumble mode & (NOTE: this is example text) This mode is used to de-tumble the spacecraft after ejection from the deployment dispenser as well as to recover it from any spin-ups. In addition to the essential components that are ON all the time, the ADCS is also operational during this mode. Any other device could be turned ON by ground command.  \\\hline
            Payload Operation Mode & This mode is used only when taking a picture.  The camera module is booted up, the camera takes a picture, stores it is RAM/ROM and then the camera is powered town again to conserve power.  This mode can be triggered by reaching a preset GPS location or manually via communications.  \\\hline
            Transmission Mode & In this mode the camera is almost constantly transmitting images taken through the camera. \\\hline
            etc. (Other Modes) &   \\\hline
        \end{tabular} } 
\end{table}

\begin{figure}[H]
    \centering
    \caption{SnapSat State Diagram}
    \vspace{0.15cm}
    Insert State diagram showing transitions between states
\end{figure}