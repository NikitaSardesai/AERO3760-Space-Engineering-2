
\section{System Budgets}

\subsection{Mass Budget} 
The mass budget is shown in table~\ref{tab:massbudget}, detailed calculations are included in Appendix~\ref{app:mass}. it is ensured that \textit{SnapSat} meets the requirement of a maximum weight of 1kg. The inertial matrix was computed using \textit{Solidworks}. It was found to be:
\begin{equation}
    I = \bmat{0.4cm}{I_{xx} * I_{xy} * I_{xz} @
                 I_{yx} * I_{yy} * I_{yz} @
                 I_{zx} * I_{zy} * I_{zz}} = 
    \bmat{0.4cm}{297788.17 * 301.92 * 5644.56 @
                 301.92 * 291591.41 * 3021.25 @
                 5644.56 * 3021.25 * 296099.96}\text{ g$\cdot$ mm}^2
\end{equation}

%\begin{landscape}
\begin{table}[H]
    \centering
    \caption{SnapSat Mass Budget (target mass of 1000 g)}
    \vspace{0.15cm}
    \label{tab:massbudget}
    {\renewcommand{\arraystretch}{1.1}%
    \begin{tabular}{|>{\arraybackslash}m{3.5cm}||>{\arraybackslash}m{2.3cm}|>{\arraybackslash}m{2.3cm}|>{\arraybackslash}m{2.3cm}|>{\arraybackslash}m{2.3cm}|}
            \hline
            {\bf Subsystem} & {\bf Mass} & {\bf Contingency} & {\bf Mass $+$ Contingency} & {\bf Fraction of Total Mass} \\ \hline\hline
            {\it Structural} \newline - chassis \newline - solar panels 
            & {\quad}\newline100g\newline $9\times25$g & {\quad}\newline30g\newline $9\times5$g & 415g & 47.6\% \\ \hline
            {\it ADCS} \newline - air core coils \newline - sun sensors \newline - IMU
            & {\quad}\newline$3\times50$g\newline $6\times1.8$g\newline 2.8g & {\quad}\newline$3\times5$g\newline $6\times0.3$g\newline0.4g & 180.2g & 22\% \\ \hline
            {\it EPS} \newline - batteries \newline - power bus
            &  {\quad}\newline 103g \newline 10g &  {\quad}\newline 7g \newline 2g & 122g & 14.9\% \\ \hline
            {\it OBS / OBDH} \newline - Arduino board \newline - memory storage \newline - PCBs
            & {\quad}\newline25g\newline1g\newline 4x10g & {\quad}\newline5g\newline 0.2g\newline2g & 79.2g & 9.1\% \\ \hline
            {\it TT\&C} \newline - antennae \newline - transceiver 
            &  {\quad}\newline 3g \newline 75g & {\quad}\newline 1g \newline 5g & 84g & 10.3\% \\ \hline
            {\it Thermal} \newline - tapings \newline - MLI
            & {\quad}\newline 1g\newline 3g & {\quad}\newline 0.3g\newline 0.5g & 4.8g & 0.6\% \\ \hline
            {\it Payload} \newline - camera
            & {\quad}\newline25g & {\quad}\newline5g & 25g & 3.1\% \\ \hline
            {\it Integration} \newline - bolts/rivets \newline - cabling/wires
            & {\quad}\newline 5g\newline2g & {\quad}\newline2g\newline0.5g & 9.5g &  1.2\%\\ \hline\hline
            {\bf Total} &  &  & 865.7g & \\ \hline
            {\bf Mass Margin} &  &  & 134.3g & 15.5\% \\ \hline
    \end{tabular} } 
\end{table} \vspace{0.3cm}
%\end{figure} 
%\end{landscape}

\subsection{Power Budget}

The power to the SnapSat will be supplied by 9 solar panels arranged around the cube as shown in Figure~\ref{fig:flower}.  This design is contingent on 3D printing the SnapSat as the solar panels will have to be depressed into the framework to allow the secondaries to fold up during transport/inside the pea pod.  In the event that it is not possible to use the fold our design, only 5 solar panels will be used and the SnapSat will transmit less and take less pictures accordingly. \\
\noindent
We will cut solar panels to 94x94 mm to produce an output of approximately 1.5W per panel.  This output will be increase again in space due to the lack of atmosphere.\\
Since we assume the satellite will spend 65\% of it's time in the sun $OAP = 0.65\cdot 1.5\cdot1.44$ \\ per panel assuming it is directly in the sun. 
By observing the layout we can assume that at any given time in the orbit 2.5 solar panels are exposed at varying angles. This gives an OAP of approximately 2.1 W, however the peak input power from the solar panels is 10.8 W (if incident light occurs from above) which the batteries are capable of charging. If cutting the solar panels does not work as expected we will purchase alternate ones and deal with the diminished power supply. \\
\noindent
This power budget is provisional and dependent on the orbital height of the SnapSat as this will greatly increase the power consumption of the S-band Tx unit. Further testing will be required to determine the exact distance the 500mW of power will propagate the signal.

\vspace{-0.3cm}
\begin{table}[H]
    \centering
    \caption{SnapSat Power Budget}
    \vspace{0.15cm}
%    \label{tab:designoverview}
    {\renewcommand{\arraystretch}{1.2}%
        \begin{tabular}{|>{\arraybackslash}m{2cm}||>{\arraybackslash}m{2cm}|>{\arraybackslash}m{2cm}|>{\arraybackslash}m{1.4cm}|>{\arraybackslash}m{1.4cm}|>{\arraybackslash}m{1.4cm}|>{\arraybackslash}m{1.4cm}|>{\arraybackslash}m{1.4cm}|}
           \hline
           \multicolumn{3}{|l|}{} & \multicolumn{5}{l|}{{\bf Average Duty Cycle by Mode (\%)}} \\ \hline
           {\bf Load} & {\bf Power Consumption (mW)} & {\bf Number of Units On} & {\it Standby Mode} & {\it Detumble Mode} & {\it Payload Operations Mode} & {\it Transmissions Mode} & {\it Power Critical Mode} \\ \hline\hline
           OBC & 55 & 1 & 25 \% & 100\%  & 100\% & 100\% & 10\% \\ \hline
           VHF Rx & 5 & 1 & 100\% & 100\% & 100\% & 100\% & 0\% \\ \hline
           S-band Tx & 500& 1 & 0\% & 5\% & 5\% & 100\% & 0\% \\ \hline
           Magnotorquers & 150 & 1 & 10 \% & 100\%  & 0\% & 0\% &  0\%\\ \hline
           Current Sensors & 140 & 1 & 100 \% & 100 \% & 100 \% & 100 \% & 100 \% \\ \hline
           Power Board & 140 & 1 & 100 \% & 100 \% & 100 \% & 100 \% & 100 \% \\ \hline
           Camera & 250 & 1 & 0\% & 0\% & 100 \% & 0\% & 0\% \\ \hline
           IMU & 30  & 1 & 25\% & 100 \% & 25\% & 25\% & 0\% \\ \hline
           Photodiodes & 120  & 1 & 25\% & 100 \% & 25\% & 25\% & 0\% \\ \hline
           GPS & 66 & 1 & 25\% & 0\% & 0\% & 0\% & 0\% \\ \hline
           \multicolumn{3}{|l|}{{\bf Sum Loads (mW)}} & 368 & 665 & 653 & 515 & 286  \\ \hline
           \multicolumn{3}{|l|}{{\bf Efficiency}} & 0.8 & 0.8 & 0.8 & 0.8 & 0.8 \\ \hline
           \multicolumn{3}{|l|}{{\bf Power Consumed (mW)}} & 450 & 831 & 816 & 644 & 357 \\ \hline
           \multicolumn{3}{|l|}{{\bf Power Generated (mW)}} & 2100 & 2100 & 2100 & 2100 & 2100 \\ \hline
           \multicolumn{3}{|l|}{{\bf Power Margin}} & 1640 & 1269 & 1284 & 1456 & 1743 \\ \hline
        \end{tabular} } 
    \end{table} \vspace{0.3cm}
    

\subsection{Pointing Budget}

Since this spacecraft is performing Earth observation, it requires a pointing budget. This refers to the ability to orient the spacecraft towards a target having a specific geographical orientation. Along with the pointing accuracy, the satellite needs to be able to map the location from its own location. Errors in both pointing and mapping accuracies will be discussed here. \\
\noindent
The attitude control system for SnapSat will consist of three air core magnetorquers operating on 3 separate planes capable of producing 0.05Am$^2$ each.  Only two of the magnetorquers can work at any one time, which will reduce total power usage for the system.  The first component of the determination system is a 9-DOF IMU which will primarily be used in the de-tumble phase due to accumulated error issues with this equipment which are expected to occur later in the mission.  The second component is a solar tracker system consisting of six photodiode pins, one on each face, which will be used to accurately determine the attitude of the satellite based on the location of the Sun. \\
\noindent
According to the specification data, the IMU will experience a 2\% error based on the expected temperature range, although this will increase over the course of the mission due to the accumulated error.  Although the exact error will need to be calculated during calibration and testing, based on current literature there are a number of similar solar tracking systems which are able to achieve an accuracy of 0.2\% \cite{beaudette}.  However given the low budget and subsequently slightly inferior equipment conservative estimate of 0.5\% will be used for the solar tracker error.  In regards to the magnetorquers expected error based on similar models 1\%, although error will be finalised during the calibration and testing phase.
\subsubsection{Error Calculation}
Due to the fact that the two attitude determination systems will almost always be used separately we have calculated three different total errors. The total errors were calculated using the following formula:
\begin{eqnarray}
\text{System Error} &=& \sqrt{(\text{IMU error})^2+(\text{sun sensor error})^2+(\text{magnetorquers error})^2} \\
&=& \sqrt{(2\%)^2+(0.5\%)^2+(1\%)^2} \\
&=& 2.3\% \\
&=& 8.3\deg
\end{eqnarray}
This is summarised in the table below

\begin{table}[H]
    \centering
    \caption{Pointing budget error calculation breakdown}
    \vspace{0.1cm}
    {\renewcommand{\arraystretch}{1.1}%
        \begin{tabular}{|>{\centering\arraybackslash}m{2.3cm}|>{\centering\arraybackslash}m{2.3cm}|>{\centering\arraybackslash}m{2.3cm}|>{\centering\arraybackslash}m{2.5cm}|>{\centering\arraybackslash}m{2cm}|>{\centering\arraybackslash}m{2cm}|}
            \hline
            & {\bf IMU Error (\%)} & {\bf Sun Tracker Error (\%)} & {\bf Magnetorquer Error (\%)} & {\bf Total (\%)} & {\bf Total (\deg)} \\ \hline\hline
            Overall System & 2.0 & 0.5 & 1.0 & 2.3 & 8.3 \\ \hline
            System 1 & 2.0 & - & 1.0 & 2.2 & 7.9 \\ \hline
            System 2 & - & 1.0 & 1.0 & 1.1 & 4.0 \\ \hline
        \end{tabular} } 
    \end{table}

\noindent
The majority of the mission is expected to be spent using system 2 (utilising the sun trackers), which produces an error of 4\deg.  Whilst this is within the range for the widest application of the three mapping scenarios (see Appendix~\ref{app:mapping}) it is slightly outside of the range of the second more focused scenario.  However, it should be noted that these are conservative calculations and the finalised error may be lower than these figures. 


 


\subsection{Link Budgets}



\subsubsection{Uplink Budget}
The results of the uplink budget (table~\ref{tab:uplink}) can be seen in the table below, where it satisfies the requirements for a successful link. The ground station utilises a cross yogi antenna for transmission into space while the satellite receives this by utilising a dipole antenna manufactured from measuring tape. The frequency band that is used for uplink is in the UHF band at 315 MHz. A low power transceiver chip was found that could receive at this frequency and also transmit at a higher frequency so the system is being based on this. This also allows the antennas to be cut to a manageable size giving less chance for a mechanical failure. The modulation utilised is Audio Frequency Shift Keying, over the audio tones of 1200Hz and 2200 Hz. With this we will be using a baud rate of 1200. AFSK is not the most efficient but it is simple and allows for less chance of failure. The expected bandwidth will fit in 20kHz at -30 dBc.

\subsubsection{Downlink Budget}
See table~\ref{tab:downlink} for the budget. As stated above the satellite will be using a dipole antenna for transmission whilst the signal will be received by a high gain antenna at the ground station.
The frequency band being utilised for downlink will be in the UHF spectrum at 433MHz, due to the data rate available and the compatibility with the chosen transceiver chip. The modulation used is Binary Phase Shift Keying, which works by modulating the phase of the reference signal. With this we will be using a baud rate of 9600 and the expected bandwidth will also fit into a 20kHz range at -30 dBc.

\begin{table}[H]
    \centering
    \caption{Uplink Budget}
    \vspace{0.2cm}
    \label{tab:uplink}
    {\renewcommand{\arraystretch}{1.2}%
        \begin{tabular}{|>{\centering\arraybackslash}m{4cm}|>{\centering\arraybackslash}m{2cm}|>{\centering\arraybackslash}m{3cm}|>{\centering\arraybackslash}m{2cm}}      
            \multicolumn{4}{l}{{\bf Information on the System}} \\
            \multicolumn{2}{l}{{\bf Transmitter: Ground Station}} & \multicolumn{2}{l}{{\bf Receiver: SnapSat}} \\
            \multicolumn{2}{l}{Orbit Altitude: 350km} & \multicolumn{2}{l}{Elevation: 30 degrees} \\
            \multicolumn{2}{l}{Slant Range: 652.5km} & \multicolumn{2}{l}{Weather: Clear Sky} \\
            \multicolumn{2}{l}{Demodulation Method: AFSK} & \multicolumn{2}{l}{Cable Length: 20m} \\
            \multicolumn{2}{l}{Antenna Type (TX): Cross Yagi} & \multicolumn{2}{l}{Antenna Type: Dipole} \\
            \multicolumn{4}{l}{{\bf Transmitter System (Ground Station)}} \\
            \multicolumn{2}{l}{\multirow{2}{*}{Ground Station Transmitter Power Output}} & \multicolumn{2}{l}{100 W} \\
            \multicolumn{2}{l}{} & \multicolumn{2}{l}{20 dBW} \\
            \multicolumn{2}{l}{Ground Station Total Transmission Line,Losses} & \multicolumn{2}{l}{3.4 dB} \\
            \multicolumn{2}{l}{Ground Station Antenna Gain} & \multicolumn{2}{l}{18.9 dBi} \\
            \multicolumn{2}{l}{Ground Station ERIP} & \multicolumn{2}{l}{35.5 dBW} \\
            \multicolumn{4}{l}{{\bf Down Link Path}} \\
            \multicolumn{2}{l}{Free-Space Path Loss} & \multicolumn{2}{l}{132 dB} \\
            \multicolumn{2}{l}{Satellite Antenna Pointing Loss (10 °)} & \multicolumn{2}{l}{10.6 dB} \\
            \multicolumn{2}{l}{Ground Station Antenna Pointing Loss (10°)} & \multicolumn{2}{l}{2.7 dB} \\
            \multicolumn{2}{l}{Satellite Transmission Line Losses} & \multicolumn{2}{l}{0.5 dB} \\
            \multicolumn{2}{l}{Atmospheric Loss (30°)} & \multicolumn{2}{l}{0.4 dB} \\
            \multicolumn{2}{l}{Ionspheric Loss} & \multicolumn{2}{l}{0.4 dB} \\
            \multicolumn{2}{l}{Rain Loss} & \multicolumn{2}{l}{0 dB} \\
            \multicolumn{2}{l}{Total Loss} & \multicolumn{2}{l}{146.6 dB} \\
            \multicolumn{4}{l}{{\bf Receiver System (on SnapSat)}} \\
            \multicolumn{2}{l}{Antenna Gain} & \multicolumn{2}{l}{2.7 dBi} \\
            \multicolumn{2}{l}{Effective Noise Temperature at Space (350K/Day)} & \multicolumn{2}{l}{1345K} \\
            \multicolumn{2}{l}{Figure of Merrit (G/Ta)} & \multicolumn{2}{l}{-28.6 dB/K} \\
            \multicolumn{2}{l}{Carrier to Thermal Noise Ratio (C/T)} & \multicolumn{2}{l}{-136.6 dB} \\
            \multicolumn{2}{l}{Boltzmann's constant (K)} & \multicolumn{2}{l}{-228 dBW/K/Hz} \\
            \multicolumn{2}{l}{Carrier to Noise Density Ratio (C/No)} & \multicolumn{2}{l}{88.9dBHz} \\
            \multicolumn{4}{l}{{\bf Modulation Process}} \\
            \multicolumn{2}{l}{System Desired Data Rate} & \multicolumn{2}{l}{1200 bps} \\
            \multicolumn{2}{l}{Demodulation Method Selected} & \multicolumn{2}{l}{AFSK} \\
            \multicolumn{2}{l}{System Allowed or Specified Bit-Error Rate} & \multicolumn{2}{l}{1.00E-04} \\
            \multicolumn{2}{l}{Demodulator Implementation Loss} & \multicolumn{2}{l}{2 dB} \\
            \multicolumn{4}{l}{{\bf Link Performance}} \\
            \multicolumn{2}{l}{Required Eb/No} & \multicolumn{2}{l}{56.1 dB} \\
            \multicolumn{2}{l}{Threshold Eb/No} & \multicolumn{2}{l}{23.2 dB} \\
            \multicolumn{2}{l}{{\bf System Link Margin}} & \multicolumn{2}{l}{{\bf 32.9 dB}}
        \end{tabular} } 
    \end{table}



\begin{table}[H]
    \centering
    \caption{Downlink Budget}
    \vspace{0.2cm}
    \label{tab:downlink}
    {\renewcommand{\arraystretch}{1.2}%
        \begin{tabular}{|>{\centering\arraybackslash}m{4cm}|>{\centering\arraybackslash}m{2cm}|>{\centering\arraybackslash}m{3cm}|>{\centering\arraybackslash}m{2cm}}      
\multicolumn{4}{l}{{\bf Information on the System}} \\
\multicolumn{2}{l}{{\bf Transmitter - SnapSat}} & \multicolumn{2}{l}{{\bf Receiver - Ground Station}} \\
\multicolumn{2}{l}{Orbit Altitude – 350km} & \multicolumn{2}{l}{Elevation – 30 degrees} \\
\multicolumn{2}{l}{Slant Range – 652.5km} & \multicolumn{2}{l}{Weather – Clear Sky} \\
\multicolumn{2}{l}{Demodulation Method – BPSK} & \multicolumn{2}{l}{Cable Length – 20m} \\
\multicolumn{2}{l}{Antenna Type (TX) –  Dipole} & \multicolumn{2}{l}{Antenna Type – Cross Yagi} \\
\multicolumn{4}{l}{{\bf Transmitter System (SnapSat)}} \\
\multicolumn{2}{l}{\multirow{2}{*}{Satellite Transmitter Power Output}} & \multicolumn{2}{l}{0.5 W} \\
\multicolumn{2}{l}{} & \multicolumn{2}{l}{-3.01 dBW} \\
\multicolumn{2}{l}{Satellite Total Transmission Line Losses} & \multicolumn{2}{l}{0.5 dB} \\
\multicolumn{2}{l}{Satellite Antenna Gain} & \multicolumn{2}{l}{2.7dBi} \\
\multicolumn{2}{l}{Satellite ERIP} & \multicolumn{2}{l}{-0.81 dBW} \\
\multicolumn{4}{l}{{\bf Down Link Path}} \\
\multicolumn{2}{l}{Free-Space Path Loss} & \multicolumn{2}{l}{132 dB} \\
\multicolumn{2}{l}{Satellite Antenna Pointing Loss (10 °)} & \multicolumn{2}{l}{10.6 dB} \\
\multicolumn{2}{l}{Ground Station Antenna Pointing Loss (10°)} & \multicolumn{2}{l}{2.7 dB} \\
\multicolumn{2}{l}{Ground Station Transmission Line Losses} & \multicolumn{2}{l}{1.8 dB} \\
\multicolumn{2}{l}{Atmospheric Loss (30°)} & \multicolumn{2}{l}{0.4 dB} \\
\multicolumn{2}{l}{Ionspheric Loss} & \multicolumn{2}{l}{0.8 dB} \\
\multicolumn{2}{l}{Rain Loss} & \multicolumn{2}{l}{0 dB} \\
\multicolumn{2}{l}{Total Loss} & \multicolumn{2}{l}{146.4 dB} \\
\multicolumn{4}{l}{{\bf Receiver System (Ground Station)}} \\
\multicolumn{2}{l}{Antenna Gain} & \multicolumn{2}{l}{14.4 dBi} \\
\multicolumn{2}{l}{Effective Noise Temperature at Sydney (350K/Day)} & \multicolumn{2}{l}{610.1 K} \\
\multicolumn{2}{l}{Figure of Merrit (G/Ta)} & \multicolumn{2}{l}{13.5 dB/K} \\
\multicolumn{2}{l}{Carrier to Thermal Noise Ratio (C/T)} & \multicolumn{2}{l}{-160.71 dB} \\
\multicolumn{2}{l}{Boltzmann's constant (K)} & \multicolumn{2}{l}{-228.6 dBW/K/Hz} \\
\multicolumn{2}{l}{Carrier to Noise Density Ratio (C/No)} & \multicolumn{2}{l}{67.89 dBHz} \\
\multicolumn{4}{l}{{\bf Modulation Process}} \\
\multicolumn{2}{l}{System Desired Data Rate} & \multicolumn{2}{l}{9600 bps} \\
\multicolumn{2}{l}{Demodulation Method Selected} & \multicolumn{2}{l}{BPSK} \\
\multicolumn{2}{l}{System Allowed or Specified Bit-Error Rate} & \multicolumn{2}{l}{1.00E-04} \\
\multicolumn{2}{l}{Demodulator Implementation Loss} & \multicolumn{2}{l}{2 dB} \\
\multicolumn{4}{l}{{\bf Link Performance}} \\
\multicolumn{2}{l}{Required Eb/No} & \multicolumn{2}{l}{26.07 dB} \\
\multicolumn{2}{l}{Threshold Eb/No} & \multicolumn{2}{l}{10.5 dB} \\
\multicolumn{2}{l}{{\bf System Link Margin}} & \multicolumn{2}{l}{{\bf 15.56 dB}}
      \end{tabular} } 
    \end{table}

\subsection{Data Budget}
The amount of data that can be transferred between the satellite and ground station is one of the major restrictions that cannot be overlooked. The low power and size of the device coupled with the large distance the data has to travel results in a small data rate. Our satellite collects information about it's position and it's surroundings  but these occupy only a small portion of the data sent back to earth. The payload is a camera which compressed JPEGs of the size 640*480. These files can range from around 20 kB up to 300 kB. Our camera will be taking photos of earth and space so the files will likely be closer to the lower end of the spectrum. \\
\\ \noindent
Although the baud rate is 9600 this is not the total amount of information we can send. We are encoding the data with the protocol AX.25 and this has about 8\% overhead. Working this through gives us 8,832 bits to use. This equates to roughly 1.08 kB/s that can be downloaded once the link has been established. After working through the link budgets and determining we would have a communication window of roughly 10 minutes, this equates to roughly 640 kBs every fly over. With a conservative estimation on the size of the pictures and factoring the system information that is included, roughly 4 pictures will be able to be downloaded per orbit.


