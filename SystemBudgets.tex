\section{System Budgets}
This section detail the power and mass budgets of SnapSat. (overview/description)


\subsection{Mass Budget} 
The mass budget is shown in table~\ref{tab:massbudget}, it is ensured that \textit{SnapSat} meets the requirement of a maximum weight of 1kg. The component masses were used to determine the center of gravity for the satellie, it is desirable to keep this located close to the geometric centre as the attitude control system (magneto-torques) are placed on the outermost surfaces. Centroid averaging across all three axes was used to calculate this as follows
\[
x_{cg} = \cfrac{\Sigma x_i\cdot m_i}{\Sigma m_i} \qquad y_{cg} = \cfrac{\Sigma y_i\cdot m_i}{\Sigma m_i} \qquad z_{cg} = \cfrac{\Sigma z_i\cdot m_i}{\Sigma m_i}
\]
\noindent
The moment of inertia about all three axes is given by
\[ I = \int r^2\,\,dm \]
\noindent
The inertial of each individual component was ignored, assuming these were roughly symmetrical. Only the relative locations of the component contributed to the total inertia. Thus, the inertias were given
\begin{eqnarray} 
I_{xx} &=& \cfrac{1}{\Sigma m_i}\,\cdot\,\left( \Sigma \sqrt{(y_i-y_{cg})^2+(z_i-z_{cg})^2}\,\cdot\,m_i \right) \\
I_{yy} &=& \cfrac{1}{\Sigma m_i}\,\cdot\,\left( \Sigma \sqrt{(x_i-x_{cg})^2+(z_i-z_{cg})^2}\,\cdot\,m_i \right) \\
I_{zz} &=& \cfrac{1}{\Sigma m_i}\,\cdot\,\left( \Sigma \sqrt{(x_i-x_{cg})^2+(y_i-y_{cg})^2}\,\cdot\,m_i \right) 
\end{eqnarray} 
\noindent
and so on; where $x_i$, $y_i$ and $z_i$ are the positions of each component and $m_i$ is the mass of each component. The inertial matrix was computed using \textit{Solidworks}. It was found to be:
\begin{equation}
    I = \bmat{0.4cm}{I_{xx} * I_{xy} * I_{xz} @
                 I_{yx} * I_{yy} * I_{yz} @
                 I_{zx} * I_{zy} * I_{zz}} =
\end{equation}

%\begin{landscape}
\begin{table}[H]
    \centering
    \caption{SnapSat Mass Budget}
    \vspace{0.15cm}
    \label{tab:massbudget}
    {\renewcommand{\arraystretch}{1.3}%
    \begin{tabular}{|>{\arraybackslash}m{3cm}||>{\arraybackslash}m{2.5cm}|>{\arraybackslash}m{2.5cm}|>{\arraybackslash}m{2.3cm}|>{\arraybackslash}m{2.3cm}|}
            \hline
            {\bf Subsystem} & {\bf Mass (g)} & {\bf Contingency (g)} & {\bf Mass and Contingency} & {\bf Fraction of Total Mass} \\ \hline\hline
            Structural &  &  &  &  \\ \hline
            ADCS &  &  &  &  \\ \hline
            EPS &  &  &  & \\ \hline
            OBS / OBDH &  &  &  & \\ \hline
            TT\&C &  &  &  &  \\ \hline
            Thermal &  &  &  &  \\ \hline
            Payload &  &  &  &  \\ \hline
            Integration &  &  &  &  \\ \hline\hline
            Total &  &  &  & \\ \hline
            Mass Margin &  &  &  &  \\ \hline
    \end{tabular} } 
\end{table} \vspace{0.3cm}
%\end{figure} 
%\end{landscape}


\subsection{Power Budget}
\vspace{-0.3cm}
\begin{table}[H]
    \centering
    \caption{SnapSat Power Budget}
    \vspace{0.15cm}
    \label{tab:designoverview}
    {\renewcommand{\arraystretch}{1.4}%
        \begin{tabular}{|>{\arraybackslash}m{2cm}||>{\arraybackslash}m{2cm}|>{\arraybackslash}m{2cm}|>{\arraybackslash}m{1.4cm}|>{\arraybackslash}m{1.4cm}|>{\arraybackslash}m{1.4cm}|>{\arraybackslash}m{1.4cm}|>{\arraybackslash}m{1.4cm}|}
           \hline
           \multicolumn{3}{|l|}{} & \multicolumn{5}{l|}{{\bf Average Duty Cycle by Mode (\%)}} \\ \hline
           {\bf Load} & {\bf Power Consumption (W)} & {\bf Number of Units On} & {\it Safe Mode} & {\it Recovery Mode} & {\it Payload Mode} & {\it Other Mode} &  \\ \hline\hline
           OBC &  &  &  &  &  &  &  \\ \hline
           VHF Rx &  &  &  &  &  &  &  \\ \hline
           S-band Tx &  &  &  &  &  &  &  \\ \hline
           Reaction Wheels &  &  &  &  &  &  &  \\ \hline
           Power Board &  &  &  &  &  &  &  \\ \hline
           Camera &  &  &  &  &  &  &  \\ \hline
           etc. &  &  &  &  &  &  &  \\ \hline
           &  &  &  &  &  &  &  \\ \hline
           &  &  &  &  &  &  &  \\ \hline
           &  &  &  &  &  &  &  \\ \hline\hline
           \multicolumn{3}{|l|}{{\bf Sum Loads (W)}} &  &  &  &  &  \\ \hline
           \multicolumn{3}{|l|}{{\bf Efficiency}} &  &  &  &  &  \\ \hline
           \multicolumn{3}{|l|}{{\bf Power Consumed (W)}} &  &  &  &  &  \\ \hline
           \multicolumn{3}{|l|}{{\bf Power Generated (W)}} &  &  &  &  &  \\ \hline
           \multicolumn{3}{|l|}{{\bf Power Margin}} &  &  &  &  &  \\ \hline
        \end{tabular} } 
    \end{table} \vspace{0.3cm}

\subsection{Pointing Budget}
Since this spacecraft is performing Earth observation, it requires a pointing budget. This refers to the ability to orient the spacecraft towards a target having a specific geographical orientation. Along with the pointing accuracy, the satellite needs to be able to map the location from its own location. Errors in both pointing and mapping accuracies will be discussed here.

\subsection{Link Budgets}
Calculations for both link budgets (list assumptions here).

\subsubsection{Uplink Budget}
The uplink budget allows for XXX. The specifications are
\begin{itemize}
    \item Antenna type at satellite: (omni, directional$+$gain)
    \item Frequency Band: (VHF (145.800MHz) , UHF (435.xxx MHz), SHF etc.)
    \item Objective C/N:
    \item Bit rate and modulation type:
    \item Expected occupied bandwidth:
\end{itemize}

\subsubsection{Downlink Budget}
The downlink budget allows for XXX. The specifications are
\begin{itemize}
    \item Antenna type at satellite: (omni, directional$+$gain)
    \item Frequency Band: (VHF (145.800MHz) , UHF (435.xxx MHz), SHF etc.)
    \item Objective C/N:
    \item Bit rate and modulation type:
    \item Expected occupied bandwidth:
\end{itemize}

\subsection{Data Budget}
Data budget CALCULATIONS.



