\section{Thermal Control Subsystem}
The method of developing thermal control used for SnapSat considers the following simplified model of the satellite. The main body is idealised as a system dissipating heat (located at the centre of the CubeSat) to the boundary located on the face of the CubeSat. This boundary is exposed to the outer environment. Energy conservation laws require that in steady state, the heat dissipated by the internal electronics is equal to that transferred tot he boundary. Thus, the heat from internal dissipation added to the heat adsorbed from the outside is equal to the heat rejected to space. The general governing equation is
\begin{equation}
    Q_{1\rightarrow2} = K_{1\rightarrow2}(T_a - T_2)
    \label{eqn:Qconduction}
\end{equation}  
\noindent
\begin{align}
    \text{Where}\quad Q &= \text{heat exchange (Watts)} \nonumber\\
    K &= \text{proportionality factor constant (Watts/Kelvin)} \nonumber\\
    T &= \text{temperature of bodies (Kelvin)} \nonumber
\end{align}
\noindent
between bodies 1 and 2. Additionally, the heat radiated from a blackbody surface of temperature $T$ is given by 
\begin{equation}
    Q_r = KT^4
    \label{eqn:Qradiation}
\end{equation} 
\noindent
Where the proportionality factor depends on physical constants, the material properties, surface conditions and geometry. A schematic of the incoming thermal radiation on the CubeSat in Low-Earth Orbit (LEO) is shown below.
\begin{figure}[H]
    \pic{0.8}{./figures/thermaloverview}{Incoming thermal radiation on the satellite}
\end{figure}
\noindent

\subsection{Total Incoming Radiation}
The total radiation incoming onto the satellite as it orbits is defined in figure~\ref{plot:incomingradiation} below. This assumes that the satellite is in full view of the sun 65\% of the time in each orbit.
\begin{figure}[H]
    \tikzpic{0.7}{0.4}{./figures/incomingradiation.tex}{Radiation incoming onto the satellite as it orbits}
    \label{plot:incomingradiation}
\end{figure}
\noindent
Whilst this is the incoming radiation on the satellite as a whole, it is not indicative of the amount of radiation received by each side of the satellite. As the spacecraft is attitude controlled, the lower side will be facing the Earth always and only receive solar radiation for a short period of time. The amount of solar radiation (and even Earth IR radiation) received depends on the projected area that the radiation falls upon. Corrections are found using the view factor of each side of the satellite.

\subsubsection{View Factors}
The view factor of each side of the satellite allows for the calculation of the effect of the incoming radiation. The calculation takes into account the projected amount of heat flux on each side. The calculation takes into account the projected 

\begin{figure}[H]
    \pic{0.55}{./figures/orbit.png}{arg3}
\end{figure}

\subsection{The Three Modes of Heat Transfer}
The first law of thermodynamics states that the internal energy change on a system is equal to the amount of heat added subtracted by the amount of work done. The work done by the satellite on its environment is zero in out case, so the change in energy becomes
\[ \cfrac{dU}{dt} = Q = A\,\rho\,c_p\,\cfrac{dT}{dt}\,dx \]
\noindent
\begin{align}
\text{Where}\quad Q &= \text{heat added (Watts)} \nonumber\\
A &= \text{cross-sectional area (m$^2$)} \nonumber\\
\rho &= \text{density of material (kg/m$^3$)} \nonumber\\
c_p &= \text{specific heat capacity (J/kg K)} \nonumber\\
T &= \text{temperature (K)} \nonumber\\
dx &= \text{incremental length (m)} \nonumber
\end{align}
Is is dependent on the physical and geometric properties of the satellite and the change in temperature. The total heat balance for the satellite is then given by the heat flux entering the system minus the flux leaving the system. These are characterised by the modes of heat transfer below.
\subsubsection{Convection}
Convection is the heat transfer between a solid surface and flowing fluid. This is of importance during mission launch, however does not apply in a space environment. Convection considerations were ignored for this design.
\subsubsection{Conduction}
Thermal energy transfer within a material due to vibrating atoms - for example if the material is heated in one location, conduction is the method by which it spreads to the rest of the material. This is most important for on-board electronics, the rate of heat transfer is given by 
\[ Q_{conduction} = \cfrac{kA}{\Delta x}\,(T_1-T_2) \]
\noindent
which is the same as equation~\ref{eqn:Qconduction}. The heat transfer depends on the area of the satellite normal to the direction of heat transfer $A$, the thermal conductivity $k$ and the temperature differential $T$.
\subsubsection{Radiation}
Perhaps the most complex form of heat transfer is radiation, where all bodies above 0K emit and absorb electromagnetic energy. We consider each body as a perfect emitter (black body) and integrate the emitted energy across all wavelengths, this gives
\begin{equation}
    E_{bb} = \epsilon\sigma\,T^4
\end{equation}
measured in Watts/m$^2$. This is the same as equation~\ref{eqn:Qradiation}, where $\sigma$ is the Stefan-Boltzmann constant. In this case, the emissivity $\epsilon$ has been added to account for the fact that the surfaces are not perfect black bodies.

\begin{figure}[H]
    \tikzpic{0.8}{0.3}{./figures/PanelIncoming3.tex}{Solar radiation falling on each panel}
\end{figure}

\begin{figure}[H]
    \tikzpic{0.8}{0.3}{./figures/PanelIncoming4.tex}{Total radiation falling upon panel 2}
\end{figure}

