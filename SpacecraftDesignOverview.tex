\section{Spacecraft Deign Overview}
Summarised in table~\ref{tab:designoverview} below is the outline of all components in the SnapSat proposed design.

\begin{table}[H]
    \centering
    \caption{SnapSat Design Overview}
    \vspace{0.15cm}
    \label{tab:designoverview}
    {\renewcommand{\arraystretch}{1.4}%
        \begin{tabular}{|>{\arraybackslash}m{3cm}|>{\arraybackslash}m{10cm}|}
            \hline
            \textbf{Subsystem} & \textbf{Description} \\ \hline\hline
            Structural & - 3D ABS printed plastic \newline - six individual sides adhered together \\\hline
            ADCS & - copper loop magneto-torgqers made in-house \newline - Osram SFH203P Photodiodes \newline - IMU: Adafruit 9-DOF  \\\hline
            EPS & - Australian Robotics solar panels \newline - battery: LiNiMnCo 26650 rechargeable cell  \\\hline
            OBC / OBDH & - Arduino DUE \newline - 4 $\times$ PCBs  \\\hline
           TT\&C & - VHF (Xbee) \newline - UHF \newline - tape measure antennae \\\hline
           Thermal & - thermal tapings and passive coatings (Kapton tape) \newline - selected components will have multi-layer insulation \\\hline
           Payload & - Leopard LI-CAM-AR0140HISPI \\\hline
        \end{tabular} } 
    \end{table}
    
\subsection{Subsystem Design Schematic}
The layout of Snapsat, with the interconnects of power and data lines between the subsystems is shown in the figure below. (NOTE: this is only an example for now)

\begin{figure}[H]
    \pic{0.5}{DesignSchematicExample.png}{Design Schematic}
\end{figure}
