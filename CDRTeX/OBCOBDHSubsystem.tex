\section{On-Board Computer and On-board Data Handling Subsystem}

The SnapSat uses an Iduino Due Pro micro controller unit (MCU) to control it's operation.  It does not use an RTOS, instead using the real time clock to trigger events at pre-set times and interrupt based control to prioritize incoming messages of the communications subsystem and other time critical systems (see figure~\ref{fig:modes}).

\subsection{Selection}

The Iduino Due Pro was chosen because it is a powerful micro controller, with a large amount of program space, sufficient peripherals to run all modules and it operates at 3.3V, which is compatible with all external circuity. Additionally it has an external removable programmer, which gives it a relatively small form factor, (86.3mm by 53.3mm) allowing to fit easily within the CubeSat. \\
The Iduino Due Pro is a derivative of the Arduino Due, so multitudes of documentation and code exist speeding up component integration.

\subsection{On-board Data Handling}

The MCU interfaces with the other modules through the SPI, I2C1, I2C2, Serial0 and Serial3 peripherals, in addition to several GPIO pins as shown in figure~\ref{fig:mcucon}.  This leaves a Serial and NMEA module free which could easily be used to accommodate further subsystems in the future.  Note that the SPI module and one of the I2C modules are shared between slave components, in such a way that more could be added to these busses in future iterations.

\begin{figure}[H]
	\pic{0.8}{./figures/mcucon}{A diagram showing the interactions between the MCU and the other subsystems.}\label{fig:mcucon}
\end{figure}

\subsection{Modes of Operation}

The CubeSat can operate in several modes of operation throughout an orbital flight, depending on the stage of flight, as shown in figure~\ref{fig:modes}.

\begin{figure}[H]
	\pic{0.8}{./figures/modes}{A diagram showing the modes of operation during flight.}\label{fig:modes}
\end{figure}

A detailed description of the modes for orbital flight is as follows: \\ \\
\noindent
\textbf{Launch Mode: } This turns the satellite off for launch to comply with CubeSat Design Specification 2.3.1. During launch the deployment switch is tripped which will turn the satellite on and transfer it into Standby Mode. \\
\noindent
\textbf{Startup Mode: }This mode is entered only when the CubeSat is first launched.  In this mode the satellite will immediately turn on the ADCS to detumble.  Once the CubeSat is sufficiently stable (not tumbling in the pitch or roll axis) or 30 minutes has elapsed (CubeSat Design Specification 2.4.2) the antennas and solar panels will be deployed. It will then move into Standby Mode. \\
\noindent
\textbf{Standby Mode }In this mode, only essential satellite systems are kept on.  This includes the OBC, EPS, VHF receiver,the GPS and intermittently the IMU.  It can transition out of Standby mode via an alert from the GPS, IMU, EPS or ground station orders.  \\
\noindent
\textbf{Payload Operation Mode: } This mode is used only when taking a picture and is entered through a location alert or ground station orders from Standby Mode or a timing alert.  The camera module is booted up, the camera takes a picture, stores it on the SD card and then the camera is powered town again to conserve power.  After finishing these tasks it will return to Standby Mode. \\
\noindent
\textbf{Transmissions Mode: } This mode is entered once communications is established with the ground station or if the GPS acknowledges that a ground station is in range.  It will relay basic telemetry and if a picture is waiting in queue it will be sent.  If the EPS detects that the power is too low it will exit Transmissions Mode, and will not allow it to enter it again until it has returned to acceptable levels.  Similarly the transmitter can be shut down by a command by the ground station.\\
\noindent
\textbf{Power Critical Mode: } In the event that the power level drops to a point where Standby is not sustainable SnapSat will enter this mode.  All systems will be turned off with the exception of the EPS for the purpose of charging the batteries.  It will exit this mode when the batteries are sufficiently charged. \\
\noindent
\textbf{De-tumble mode: } This mode is used to de-tumble the spacecraft after deplopment into orbit as well as to recover it from any spin states (such as after Safe Mode). All Safe Mode components are ON, as well as the ADCS system. Other devices can be turned ON by ground command without a change of state. \\

Balloon flight, unlike orbital flights, no not need to adhere to CubeSat Design Specifications regarding power upon launch~\cite{cds}. Consequently, SnapSat is to be launched in Standby mode while moving into Payload Operations Mode and Transmissions Mode as required.  Additionally, Power Critical Mode and all power monitoring inside modes are not enabled as their purpose is to conserve power and increasing charging capacity, so without input power from solar panels they are ineffectual.